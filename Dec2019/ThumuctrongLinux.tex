\documentclass[twoside]{article}

\usepackage{amssymb}
\usepackage{hyperref} % For hyperlinks in the PDF
\usepackage[utf8]{vietnam}
\usepackage[left=2cm,right=2cm,top=2cm,bottom=2cm]{geometry}
\usepackage{footnote}
\usepackage{array}
\renewcommand\thesubsection{\arabic{subsection}}
\setlength{\parindent}{0pt}
\renewcommand\thesection{\Roman{section}}
\renewcommand\thesubsection{\arabic{subsection}}
%----------------------------------------------------------------------------------------
%	TITLE SECTION
%----------------------------------------------------------------------------------------

\title{CẤU TRÚC VÀ Ý NGHĨA CÂY THƯ MỤC TRONG LINUX} % Article title
\author{%
\textsc{Hiep Le Tuan} % Your name
\normalsize Sun* Viet Nam \\ % Your institution
\normalsize \href{mailto:le.tuan.hiep@sun-asterisk.com}{le.tuan.hiep@sun-asterisk.com} 
}
\date{\today} % Leave empty to omit a date

\begin{document}
%----------------------------------------------------------------------------------------
%	TITLE SECTION
%----------------------------------------------------------------------------------------
\maketitle
% Print the title

%----------------------------------------------------------------------------------------
%	ARTICLE CONTENTS
%----------------------------------------------------------------------------------------
\section{Cấu trúc cây thư mục trong linux}
\textbf{Bạn có bao giờ thắc mắc tại sao các một số chương trình trong Linux được lưu dưới các thư mục khác nhau như /bin, /sbin, /usr/bin hay /usr/sbin?}
Ví dụ như, một số được lưu trong /usr/bin. Sao không là /bin hay /sbin? Điểm khác biệt giữa các thư mục đó là gì? \\
Trong bài viết này, chúng tôi sẽ ôn lại giúp bạn về cấu trúc file hệ thống của Linux và ý nghĩa của từng thư mục chính.
	\begin{enumerate}
		\item \textbf{root} \\
		Đúng với tên gọi của mình: nút gốc (root) đây là nơi bắt đầu của tất cả các file và thư mục. Chỉ có root user mới có quyền ghi trong thư mục này. Chú ý rằng /root là thư mục home của root user chứ không phải là /.
		\item \textbf{/bin – Chương trình của người dùng} \\
		Thư mục này chứa các chương trình thực thi. Các chương trình chung của Linux được sử dụng bởi tất cả người dùng được lưu ở đây. Ví dụ như: ps, ls, ping…
		\item \textbf{/sbin – Chương trình hệ thống}
		Cũng giống như /bin, /sbinn cũng chứa các chương trình thực thi, nhưng chúng là những chương trình của admin, dành cho việc bảo trì hệ thống. Ví dụ như: reboot, fdisk, iptables…
		\item \textbf{/etc – Các file cấu hình}
		Thư mục này chứa các file cấu hình của các chương trình, đồng thời nó còn chứa các shell script dùng để khởi động hoặc tắt các chương trình khác. Ví dụ: /etc/resolv.conf, /etc/logrolate.conf
		\item \textbf{/dev – Các file thiết bị}
		Các phân vùng ổ cứng, thiết bị ngoại vi như USB, ổ đĩa cắm ngoài, hay bất cứ thiết bị nào gắn kèm vào hệ thống đều được lưu ở đây. Ví dụ: /dev/sdb1 là tên của USB bạn vừa cắm vào máy, để mở được USB này bạn cần sử dụng lệnh mount với quyền root: mount /dev/sdb1 /tmp
		\item \textbf{/tmp – Các file tạm}
		Thư mục này chứa các file tạm thời được tạo bởi hệ thống và các người dùng. Các file lưu trong thư mục này sẽ bị xóa khi hệ thống khởi động lại.
		\item \textbf{/proc – Thông tin về các tiến trình}
		Thông tin về các tiến trình đang chạy sẽ được lưu trong /proc dưới dạng một hệ thống file thư mục mô phỏng. Ví dụ thư mục con /proc/{pid} chứa các thông tin về tiến trình có ID là pid (pid ~ process ID). Ngoài ra đây cũng là nơi lưu thông tin về về các tài nguyên đang sử dụng của hệ thống như: /proc/version, /proc/uptime…
		\item \textbf{/var – File về biến của chương trình}
		Thông tin về các biến của hệ thống được lưu trong thư mục này. Như thông tin về log file: /var/log, các gói và cơ sở dữ liệu /var/lib…
		\item \textbf{/usr – Chương trình của người dùng}
		Chứa các thư viện, file thực thi, tài liệu hướng dẫn và mã nguồn cho chương trình chạy ở level 2 của hệ thống. Trong đó:
		\begin{itemize}
			\item /usr/bin chứa các file thực thi của người dùng như: at, awk, cc, less… Nếu bạn không tìm thấy chúng trong /bin hãy tìm trong /usr/bin
			\item /usr/sbin chứa các file thực thi của hệ thống dưới quyền của admin như: atd, cron, sshd… Nếu bạn không tìm thấy chúng trong /sbin thì hãy tìm trong thư mục này.
			\item /usr/lib chứa các thư viện cho các chương trình trong /usr/bin và /usr/sbin
			\item /usr/local chứa các chương tình của người dùng được cài từ mã nguồn. Ví dụ như bạn cài apache từ mã nguồn, nó sẽ được lưu dưới /usr/local/apache2
		\end{itemize}
		\item \textbf{/home – Thư mục người của dùng}
		Thư mục này chứa tất cả các file cá nhân của từng người dùng. Ví dụ: /home/john, /home/marie
		\item \textbf{/boot – Các file khởi động}
		Tất cả các file yêu cầu khi khởi động như initrd, vmlinux. grub được lưu tại đây. Ví dụ vmlixuz-2.6.32-24-generic
		\item \textbf{/lib – Thư viện hệ thống}
		Chứa cá thư viện hỗ trợ cho các file thực thi trong /bin và /sbin. Các thư viện này thường có tên bắt đầu bằng ld* hoặc lib*.so.*. Ví dụ như ld-2.11.1.so hay libncurses.so.5.7
		\item \textbf{/opt – Các ứng dụng phụ tùy chọn}
		Tên thư mục này nghĩa là optional (tùy chọn), nó chứa các ứng dụng thêm vào từ các nhà cung cấp độc lập khác. Các ứng dụng này có thể được cài ở /opt hoặc một thư mục con của /opt
		\item \textbf{/mnt – Thư mục để mount}
		Đây là thư mục tạm để mount các file hệ thống
		\item \textbf{/media – Các thiết bị gắn có thể gỡ bỏ}
		Thư mục tạm này chứa các thiết bị như CdRom /media/cdrom. floppy /media/floopy hay các phân vùng đĩa cứng /media/Data (hiểu như là ổ D:/Data trong Windows)
		\item \textbf{/srv – Dữ liệu của các dịch vụ khác}
		Chứa dữ liệu liên quan đến các dịch vụ máy chủ như /srv/svs, chứa các dữ liệu liên quan đến CVS.
	\end{enumerate}
\section{Ý NGHĨA CỦA CÁC THƯ MỤC TRONG LINUX}
\subsection{Ý nghĩa từng thư mục trong root directory}
\textbf{Trong Linux không có khái niệm ổ đĩa như Windows. Nếu muốn sử dụng thì phải dùng lệnh để mount.}\\
Trong hệ điều hành Linux, những tập tin mà người sử dụng nhìn thấy được đều theo cấu trúc cây thư mục, với root nằm ở trên cùng. Từ điểm này các thư mục và tập tin mới mọc nhánh ra lan dần xuống phía dưới. Thư mục cao nhất, được ký hiệu bằng vạch /, được gọi là root directory (thư mục gốc). \\
Với người sử dụng bình thường thì cây thư mục này là một dải những tập tin và thư mục nối liền nhau. Trên thực tế, nhiều thư mục trong cây thư mục này nằm ở nhiều vị trí vật lý khác nhau, trên các partition khác nhau, và thậm chí trên các ổ đĩa khác nhau. Khi một trong các partition ấy được kết nối với cấu trúc cây tại một thư mục gọi là mount point (điểm kết nối, điểm lắp ráp), thì mount point này và tất cả các thư mục cấp dưới được gọi là file system. \\
Hệ điều hành Linux hình thành từ nhiều thư mục và tập tin khác nhau. Các thư mục có thể lập thành nhiều file system khác nhau, tùy vào cách cài đặt bạn đã chọn. Nhìn chung, đa phần hệ điều hành nằm ở hai file system: root file system (file system gốc) được ký hiệu là /, và một file system khác được kết nối theo /usr (đọc là user).\\

Khi dùng lệnh cd / để chuyển về thư mục gốc và gọi hiển thị danh sách thư mục, bạn sẽ thấy nhiều thư muc. Những thư mục này tạo thành nội dung của root file system, đồng thời cung cấp mount point cho các file system khác. \\

Thư mục /bin chứa các chương trình thi hành được, còn gọi là các binaries (nhị phân). Chúng là chương trình hệ thống chủ yếu. \\

Nhiều lệnh của Linux, chẳng hạn như ls, là các chương trình nằm tại các thư mục ấy.Thư mục /sbin chứa các file nhị phân hệ thống. Hầu hết các tập tin ở đây dùng để quản trị hệ thống. (superuser-bin). \\

Thư mục /etc rất quan trọng vì chứa nhiều file cấu hình Linux. Chúng giúp cho hệ thống máy bạn có “cá tính”. File mật khẩu passwd nằm ở thư mục này, cũng như fstab , danh sách các file system cần nạp vào khi khởi động máy. \\

Ngoài ra thư mục còn chứa các script khởi động cho Linux, danh sách các host kèm địa chỉ IP, cùng với nhiều thông tin cấu hình khác.\\

Các thư viện dùng chung được chứa trong thư mục /lib. Khi dùng chung thư viện, nhiều chương trình sẽ sử dụng lại cùng loại mã, hơn nữa khi được chứa cùng chỗ, thư viện sẽ giúp giảm thiểu kích cỡ chương trình ở khía cạnh thời gian chạy. \\

Thư mục /dev chứa các file đặc biệt gọi là device files (file thiết bị, được hệ thống sử dụng để chạy các phần cứng. Ví dụ file/dev/mouse sẽ đọc thông tin từ chuột. Khi tổ chức sử dụng phần cứng theo cách này, Linux làm cho việc tương tác với phần cứng trông giống như một phần mềm. Điều này có nghĩa là trong nhiều trường hợp, bạn có thể dùng cú pháp như khi dùng với directory của mình trên đĩa mềm, ta có thể dùng lệnh: tar -cdf /dev/fd0 tackett . \\

/dev/fd0 chỉ cho lệnh tar biết phải dùng đĩa mềm. Nhiều thiết bị trong thư mục /dev được tập hợp thành nhóm logical. \\
\begin{savenotes}
    \begin{table}[ht]
    	\centering
        \begin{tabular}{|m{5em}|m{35em}|} 
            \hline
            File thiết bị & Description\\
            \hline
            /dev/console & Bàn giao tiếp hệ thống, là màn hình nối kết vật lý với hệ thống\\
            \hline
            /dev/hd* & Giao diện driver cho các ổ cứng IDE. Thiết bị /dev/hda1 chỉ partition đầu tiên trên ổ cứng had. Thiết bị /dev/had chỉ toàn bộ ổ cứng hda.\\
            \hline
            /dev/sd* & Giao diện driver cho các ổ đĩa SCSI. Những ổ đĩa và partition này có cùng quy ước với thiết bị IDE /dev/hd*. \\
            \hline
            /dev/fd* & Driver thiết bị hỗ trợ đĩa mềm. Ổ đĩa mềm đầu tiên là /dev/fd0, ổ thứ hai là /dev/fd1. \\
            \hline
            /dev/st* & Driver thiết bị cho ổ cứng băng từ SCSI.\\
            \hline
            /dev/tty* & Driver cung cấp nhiều loại thiết bị giao tiếp khác nhau cho user nhập liệu. Sở dĩ viết tắt là tty bởi vì trước kia các terminal dạng teletype đều móc nối với hệ điều hành UNIX. Với Linux, những tập tin này hỗ trợ các thiết bị giao tiếp ảo, mà bạn có thể truy cập bằng cách bấm từ cho đến . Thiết bị giao tiếp ảo cho phép nhiều user đăng nhập cùng lúc.\\
            \hline 
            /dev/pty* & Driver hỗ trợ terminal giả, dùng cho việc đăng nhập từ xa, chẳng hạn như những phiên đăng nhập qua Telnet.\\
			\hline            
            /dev/ttyS* & Các cổng giao diện nối tiếp trên máy bạn. File /dev/ttyS0 tương ứng COM1 của MS-DOS. Nếu bạn sử dụng chuột nối tiếp, thì file /dev/mouse là một liên kết tượng trưng, nối với thiết bị ttyS tương ứng (Chuột nối kết với thiết bị này.) \\
            \hline
            /dev/cua* & Các thiết bị đặc biệt gọi ra ngoài dùng với modem\\
            \hline
            /dev/null & Một thiết bị rất đặc biệt, chủ yếu là một lỗ đen. Tất cả các dữ liệu ghi vào /dev/null xem như bị mất vĩnh viễn. Việc này hữu ích khi bạn muốn chạy một câu lệnh và thủ tiêu stdout hoặc stderr. Và nếu /dev/null dùng làm file nhập, bạn sẽ tạo ra một file có độ dài zero.\\
            \hline
        \end{tabular}
        \caption{Các thiết bị thường dùng chứa trong thư mục /dev}
    \end{table}%
\end{savenotes}

Thư mục /proc là một file system ảo, dùng để đọc thông tin từ bộ nhớ.\\

Thư mục /tmp chứa các file tạm mà chương trình tạo ra trong khi chạy. Nếu bạn biết hệ thống máy mình có chương trình tạo ra nhiều file tạm với kích cỡ lớn, bạn nên tạo thư mục /tmp thành một file system riêng thay vì đặt nó vào file system gốc như là một thư mục bình thường. Bởi vì với đà chất chứa các file tạm ngày càng nhiều, file system gốc sẽ nhanh chóng bị đầy. \\

Thư mục /home là thư mục cơ sở của các home directory cho các user. Quản trị viên thường đặt /home thành file system riêng rẽ nhằm tạo nhiều khoảng trống cho user sử dụng. Ngoài ra nếu hệ thống máy bạn có nhiều user, bạn nên chia thư mục /home thành nhiều file system khác nhau. Ví dụ bạn có thể tạo ra /home/gocit cho các thành viên của nhóm điều hành của công ty và/home/admin cho quản trị viên. Mỗi thư mục như thế sẽ là một file system riêng, bên dưới có home directory riêng cho các user tương ứng.\\

Thư mục /var lưu các file có thể thay đổi kích thước theo thời gian. Nhiều file đăng nhập hệ thống (system log file) thường nằm trong thư mục này. Thư mục /var/spool cùng với các thư mục con dùng để chứa dữ liệu như tin tức hoặc thư tín mới nhận được, hoặc giả đang chờ gửi đi nơi khác. \\

Thư mục /usr và các thư mục con rất quan trọng cho hệ thống Linux, bởi vì chứa đựng nhiều thư mục trong đó có những chương trình cần thiết nhất cho hệ thống. Những thư mục cấp dưới của /usr chứa các gói phần mềm lớn mà bạn đã cài đặt. \\

	\begin{table}[ht]
    	\centering
        \begin{tabular}{|m{5em}|m{35em}|} 
        	\hline
        	Thư mục thứ cấp & Desciption \\
        	\hline
        	/usr/bin & Lưu nhiều file thi hành của hệ thống.\\
        	\hline
        	/usr/etc & Lưu nhiều file cấu hình hệ thống \\
        	\hline
        	/usr/include & Tại đây và trong nhiều thư mục cấp dưới của /usr/include là nơi lưu tất cả các file kèm theo bộ biên dịch C. Những file header này định nghĩa các hằng và hàm dùng trong lập trình bằng C. \\
        	\hline
        	/usr/g++-include & Lưu các file kèm theo bộ biên dịch C.\\
        	\hline
        	/usr/lib &  Chứa các thư viện để chương trình sử dụng trong khi kết nối \\
        	\hline
        	/usr/share/man & Chứa các trang thủ công cho chương trình. Bên dưới /usr/share/man là nhiều thư mục tương ứng với các đoạn trong trang man.\\
        	\hline
        	/dev/pty* & Driver hỗ trợ terminal giả, dùng cho việc đăng nhập từ xa, chẳng hạn như những phiên đăng nhập qua Telnet.\\
        	\hline
        	/usr/src & Chứa các thư mục mã nguồn của nhiều chương trình trên hệ thống. Nếu nhận được gói phần mềm chờ cài đặt, bạn nên lưu vào /usr/src/tên-gói trước khi cài đặt.\\
        	\hline
        	/usr/local & Dành riêng cho việc thiết kế hoặc tùy chỉnh các ứng dụng cho phù hợp với hệ thống máy bạn. Nhìn chung, hầu hết phần mềm dùng tại chỗ được lưu trong các thư mục cấp dưới của thư mục này. \\
        	\hline
        \end{tabular}
        \caption{Các thư mục thứ cấp quan trọng trong file system /usr.}
	\end{table}
	
\subsection{Tóm tắt mục đích sử dụng từng thư mục}
\textbf{Các thư mục trong Linux được chuẩn hóa và mỗi thư mục có một mục đích sử dụng nhất định:} \\
/bin – chứa các ứng dụng quan trọng (binary applications).\\
/boot – các tập tin cấu hình cho quá trình khởi động hệ thống (boot configuration files) \\
/dev – chứa các tập tin là chứng nhận cho các thiết bị của hệ thống (device files)\\
/etc – chứa các tập tin cấu hình của hệ thống, các tập tin lệnh để khởi động các dịch vụ của hệ thống…\\
/home – thư mục này chứa các thư mục cá nhân của những người có quyền truy cập vào hệ thống (local users’ home directories)\\
/lib – thư mục này lưu các thư viện chia sẻ của hệ thống (system libraries)\\
/lost+found – thư mục này được dùng để lưu các tập tin không có thư mục mẹ mà được tìm thấy dưới thư mục gốc (/) sau khi thực hiện lệnh kiểm tra hệ thống tập tin (fsck).\\
/media – thư mục này được dùng để tạo ra các tập tin gắn (loaded) tạm thời được hệ thống tạo ra khi một thiết bị lưu động (removable media) được cắm vào như đĩa CDs, máy ảnh kỹ thuật số…\\
/mnt – thư mục này được dùng để gắn các hệ thống tập tin tạm thời (mounted filesystems),\\
/opt – thư mục dùng dể chứa các phần mềm ứng dụng (optional applications) đã được cài đặt thêm.\\
/proc – đây là một thư mục đặc biệt linh động để lưu các thông tin về tình trạng của hệ thống, đặc biệt về các tiến trình (processes) đang hoạt động.\\
/root – đây là thư mục nhà của người quản trị hệ thống (root). \\
/sbin – thư mục này lưu lại các tập tin thực thi của hệ thống (system binaries)\\
/sys – thư mục này lưu các tập tin của hệ thống (system files).\\
/tmp – thư mục này lưu lại các tập tin được tạo ra tạm thời (temporary files).\\
/usr – thư mục này lưu và chứa những tập tin của các ứng dụng chính đã được cài đặt cho mọi người dùng (all users). \\
/var – thư mục này lưu lại tập tin ghi các số liệu biến đổi (variable files) như các tập tin dữ liệu và tập tin bản ghi \\
%----------------------------------------------------------------------------
%	REFERENCES
%----------------------------------------------------------------------------
\begin{thebibliography}{xxx}
\bibitem{}
Evi Nemeth, Garth Snyder, Trend R.Hein, Ben Whaley(2011), ‘‘UNIX and Linux System Administration Handbook, 4th Edition’’, \emph{CNIKM'09}, \textbf{15}
(1),
\end{thebibliography}
\end{document}
